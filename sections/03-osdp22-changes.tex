\section{Changes to OSDP 2.2}\label{sec:osdp22-changes}

This section documents the normative deltas that apply when an OSDP~2.2 implementation adopts the Enhanced PIV profile. Each subsection references the original command definition and records the precise behavioral updates, encoding clarifications, and new error handling rules introduced by this proposal.

\subsection{Updated Command Definitions}

\subsubsection{Authentication Challenge (osdp\_CRAUTH)}

OSDP~2.2 transports the ISO/IEC~7816-4 \texttt{GENERAL AUTHENTICATE} command through \texttt{osdp\_CRAUTH}. The OSDP~2.3 profile maintains the same framing fields (\texttt{TOTAL}, \texttt{OFFSET}, \texttt{DATA\_LEN}) but extends the semantics of the command payload so PDs can carry both the classic RSA signing flow and the enhanced elliptic-curve key-agreement flow.

\begin{itemize}
  \item The \emph{Algorithm} field continues to be sent in the first fragment only. Codes \texttt{0x05}, \texttt{0x06}, and \texttt{0x07} select RSA signing operations and preserve backwards compatibility with OSDP~2.2. Codes \texttt{0x08}, \texttt{0x0A}, and \texttt{0x0C} select AES encryption using 128-, 192-, and 256-bit keys respectively. Codes \texttt{0x11} and \texttt{0x14} select elliptic-curve key agreement using the P-256 and P-384 curves.
  \item The \emph{Key} field retains its role as the card’s reference identifier. PDs \textbf{SHALL NOT} reinterpret or remap the identifier.
  \item For RSA algorithms (\texttt{0x05}, \texttt{0x06}, \texttt{0x07}), the \emph{Challenge} field carries the datum to be signed. The PD forwards the value unmodified; padding (for example PKCS\#1 v1.5 or PSS) is the responsibility of the ACU.
  \item For AES algorithms (\texttt{0x08}, \texttt{0x0A}, \texttt{0x0C}), the \emph{Challenge} field contains the plaintext block(s) to be encrypted. PDs \textbf{SHALL} send the data to the credential without padding; ACUs \textbf{SHALL} provide data aligned to the block size dictated by the selected key length and card profile.
  \item For algorithms \texttt{0x11} and \texttt{0x14}, the \emph{Challenge} field conveys the peer public key. The PD \textbf{SHALL} issue \texttt{GENERAL AUTHENTICATE} with \texttt{CLA = 0x00} (or \texttt{0x0C} when a secure channel is active), \texttt{INS = 0x87}, \texttt{P1 = Algorithm}, and \texttt{P2 = Reference Identifier}. The command data \textbf{SHALL} be encoded as \texttt{7C Len 82 00 85 Len (04 || X || Y)}, where \texttt{X} and \texttt{Y} are the uncompressed coordinates for the selected curve and are left-padded with \texttt{0x00} as required. The credential performs elliptic-curve Diffie–Hellman in accordance with SP~800-73; the derived secret is returned in the reply fragments described below.
\end{itemize}

\begin{PacketTable}{Authentication Challenge (osdp\_CRAUTH)}
  \PacketRow{1}{CMND}{Command identifier.}{0xA5}
  \PacketRow{2}{TOTAL}{Length of the complete message, least-significant byte first.}{0x0000--0xFFFF}
  \PacketRow{2}{OFFSET}{Offset of this fragment within the complete message, least-significant byte first.}{0x0000--0xFFFF}
  \PacketRow{2}{DATA\_LEN}{Length of the fragment payload, least-significant byte first.}{0x0000--0xFFFF}
  \PacketRow{1}{Algorithm}{Cryptographic mechanism. Supported values include \texttt{0x05}/\texttt{0x06}/\texttt{0x07} (RSA signature), \texttt{0x08}/\texttt{0x0A}/\texttt{0x0C} (AES encryption with 128/192/256-bit keys), and \texttt{0x11}/\texttt{0x14} (ECC-P256/P384 ECDH).}{0x00--0xFF}
  \PacketRow{1}{Reference Identifier}{Key reference passed to \texttt{GENERAL AUTHENTICATE}.}{0x00--0xFF}
  \PacketRow{0--n}{Challenge}{For RSA algorithms, the datum to be signed (no padding applied by the PD). For AES algorithms, the plaintext block(s) to be encrypted, pre-padded by the ACU as required. For \texttt{0x11}/\texttt{0x14}, the peer public key encoded as \texttt{7C Len 82 00 85 Len (04 || X || Y)} with coordinates padded to curve length.}{0x00--0xFF}
\end{PacketTable}


\subsubsection{Response to Challenge (osdp\_CRAUTHR)}

The reply continues to use the OSDP~2.2 multi-part format. Successful execution yields a response containing the credential’s entire \texttt{GENERAL AUTHENTICATE} data (minus the ISO/IEC~7816 status word); PDs \textbf{SHALL NOT} strip outer BER-TLV tags or lengths. If the credential reports an error, the PD \textbf{SHALL} return \texttt{osdp\_NAK} using the appropriate code from the standard set (for example \texttt{0x01} security condition not met, \texttt{0x02} card removed, \texttt{0x03} PIN blocked, \texttt{0x04} wrong interface, \texttt{0x05} incorrect parameter, \texttt{0x06} PIN verification failed, \texttt{0x07} PIN verification timeout, \texttt{0x09} unsupported algorithm/reference, \texttt{0x0A} general authentication failure).

\begin{PacketTable}{Response to Challenge (osdp\_CRAUTHR)}
  \PacketRow{1}{CMND}{Reply identifier.}{0x82}
  \PacketRow{2}{TOTAL}{Length of the complete response, least-significant byte first.}{0x0000--0xFFFF}
  \PacketRow{2}{OFFSET}{Offset of this fragment within the complete response, least-significant byte first.}{0x0000--0xFFFF}
  \PacketRow{2}{DATA\_LEN}{Length of the fragment payload, least-significant byte first.}{0x0000--0xFFFF}
  \PacketRow{0--n}{Response Data}{Signature, encrypted block, or shared-secret material returned by the credential. The credential’s BER-TLV structure is preserved exactly as received (status word excluded).}{0x00--0xFF}
\end{PacketTable}


\subsubsection{Get PIV Data (osdp\_PIVDATA)}

OSDP~2.2 defined \texttt{osdp\_PIVDATA} with a five-byte payload that selected a PIV object and optional element. The original text did not explicitly specify how to pad identifiers shorter than three bytes, how nested tags should be returned, or how to interpret the data offset. OSDP~2.3 clarifies these behaviors so that ACUs and PDs apply a single normative approach.

\begin{itemize}
  \item The ACU \textbf{SHALL} transmit the PIV Object Identifier as three bytes encoded most-significant byte first. When the identifier contains fewer than three significant bytes, the ACU \textbf{SHALL} left-pad the value with \texttt{0x00} or \texttt{0x0000}, as required, so that the transmitted field is exactly three bytes. The PD \textbf{SHALL NOT} forward those pad bytes to the credential; it \textbf{SHALL} issue the ISO~7816-4 \texttt{GET DATA} APDU using only the significant identifier bytes. If VCI secure messaging is established, the PD \textbf{SHALL} use the appropriate secure channel automatically; no explicit selector is present in the command.
  \item The field previously described as “PIV element ID” is renamed \emph{PIV Tag}. A tag value of \texttt{0x00} directs the PD to return the complete BER-TLV structure, including the outer \texttt{0x53} or \texttt{0x73} tag and length. Any non-zero value selects a single root-level tag within the object. The PD \textbf{MUST} remove the outer wrapper only when it is \texttt{0x53} or \texttt{0x73}; if the outer wrapper has any other value, the PD \textbf{MUST} reject the command unless the tag value is \texttt{0x00}. Tags \textbf{SHALL} be a single byte in the range \texttt{0x01}--\texttt{0xFF}; multi-byte DER tags are not supported and \textbf{MUST} be retrieved by issuing a tag value of \texttt{0x00} and parsing the returned object at the ACU.
  \item The data offset now supports one- or two-byte encodings. PDs \textbf{SHALL} accept either form and determine which was used based on the command length. The offset applies to the selected payload (entire object when the tag is zero, or the chosen tag otherwise). PDs \textbf{SHALL} return \texttt{osdp\_NAK} with error code \texttt{0x0D} when security prerequisites are not satisfied and \texttt{0x0E} when the requested object or tag is absent; see Subsection~\ref{sec:new-piv-error-codes}.
\end{itemize}

\begin{PacketTable}{Get PIV Data (osdp\_PIVDATA)}
  \PacketRow{1}{CMND}{Command identifier.}{0xA3}
  \PacketRow{3}{PIV Object Identifier}{Three-byte identifier of the PIV data object per SP~800-73-4, encoded MSB first. Leading zero bytes SHALL be supplied by the ACU when the identifier is shorter than three bytes; the PD SHALL drop those pad bytes before issuing the APDU.}{0x00--0xFF\textsuperscript{1}}
  \PacketRow{1}{PIV Tag}{Single-byte ASN.1 DER tag of a root-level element. Tag value 0x00 selects the entire object; non-zero tags select a root-level element and require the PD to strip the outer 0x53/0x73 wrapper, verify the length, and return only the tagged value. Multi-byte tags are not directly addressable and SHALL be retrieved by requesting tag 0x00.}{0x00; or 0x01--0xFF}
  \PacketRow{1--2}{Data Offset}{Byte offset within the requested content. Single-byte and two-byte encodings SHALL be accepted; the PD determines the format from the packet length.}{0x00--0xFFFF}
\end{PacketTable}

\begin{flushleft}
\footnotesize\textsuperscript{1}\,Each position MAY carry any value from \texttt{0x00} to \texttt{0xFF}; the ACU pads on the left with \texttt{0x00} when the identifier has fewer than three significant bytes.
\end{flushleft}


\subsubsection{PIV Data Reply (osdp\_PIVDATAR)}

The reply structure is unchanged from a field-count perspective but 2.3 aligns the semantics with the command refinements above:

\begin{itemize}
  \item \texttt{TOTAL}, \texttt{OFFSET}, and \texttt{DATA\_LEN} continue to be encoded LSB first; they now explicitly report the size, fragment base, and fragment length for the content selected by the \texttt{osdp\_PIVDATA} request (entire object or chosen tag).
  \item \texttt{CARD\_DATA} \textbf{SHALL} convey the exact bytes obtained from the credential after applying any tag stripping specified by the command. No additional framing or padding is permitted.
\end{itemize}

\begin{PacketTable}{PIV Data Reply (osdp\_PIVDATAR)}
  \PacketRow{1}{CMND}{Reply identifier.}{0x80}
  \PacketRow{2}{Total Length}{Length in bytes of the complete data object or tagged element, least-significant byte first.}{0x0000--0xFFFF}
  \PacketRow{2}{Fragment Offset}{Offset in bytes of this fragment within the complete payload, least-significant byte first.}{0x0000--0xFFFF}
  \PacketRow{2}{Fragment Length}{Length in bytes of the data carried in this fragment, least-significant byte first.}{0x0000--0xFFFF}
  \PacketRow{0--n}{Card Data}{Requested data bytes exactly as returned by the credential. When a tagged element was selected, this sequence contains only the element’s BER-TLV contents, excluding the selected tag and length.}{0x00--0xFF}
\end{PacketTable}


\subsection{New Error Codes for Enhanced PIV}\label{sec:new-piv-error-codes}

The following \texttt{osdp\_NAK} error codes are introduced by the Enhanced PIV profile. They extend the set defined in OSDP~2.2 without reassigning existing values and are available to any command that requires them.

\begin{samepage}
\begin{center}
  \captionsetup{type=table}
  \captionof{table}{Enhanced PIV Error Codes}
  \begin{tabular}{|c|p{9cm}|}
    \hline
    \rowcolor{PacketHeaderBg}
    \textcolor{PacketHeaderFg}{\bfseries Code} & \textcolor{PacketHeaderFg}{\bfseries Meaning} \\\hline
    0x0A & Function not supported. \\\hline
    0x0B & Insufficient PD memory to complete the requested operation. \\\hline
    0x0C & Insufficient credential memory or storage space. \\\hline
    0x0D & Security status not satisfied. \\\hline
    0x0E & Data object or tag not found. \\\hline
    0x0F & VCI not established. \\\hline
    0x10 & PIV Auto not configured. \\\hline
  \end{tabular}
\end{center}
\end{samepage}

\subsection{Updated Function Codes}

\subsubsection{Function Code 13 – Reader Interfaces}

OSDP~2.2 defined Function Code~13 as a reader count indicator with a compliance byte fixed at \texttt{0x00}. Enhanced PIV-capable PDs \textbf{SHALL} continue to populate the reader count field, but they MAY set bits within the compliance byte to advertise the downstream credential technologies they expose. A value of \texttt{0x00} remains valid and retains the legacy meaning of “interface presence unspecified.”

When a PD reports specific interfaces, it \textbf{SHALL} encode the compliance byte using the flags in Table~\ref{tab:function-code-13-flags}. The reader count byte continues to report the number of attached credential interfaces.

\begin{samepage}
\begin{center}
  \captionsetup{type=table}
  \captionof{table}{Function Code 13 Compliance Flags}
  \label{tab:function-code-13-flags}
  \begin{tabular}{|c|p{9cm}|}
    \hline
    \rowcolor{PacketHeaderBg}
    \textcolor{PacketHeaderFg}{\bfseries Mask} & \textcolor{PacketHeaderFg}{\bfseries Meaning} \\\hline
    0x01 & PD presents a contact card interface (ISO~7816 or equivalent). \\\hline
    0x02 & PD presents a high-frequency contactless credential interface (13.56~MHz NFC). \\\hline
    0x04 & PD supports the OSDP~2.3 Enhanced PIV command set (including \texttt{osdp\_CRAUTH}, \texttt{osdp\_PIVDATA}, \texttt{osdp\_CARDSTATUS}, and \texttt{osdp\_PIVMODE}). \\\hline
    0x08 & PD presents a low-frequency credential interface (approximately 125~kHz prox). \\\hline
    0x10 & PD presents a Bluetooth credential interface. \\\hline
    0x20 & PD presents a barcode credential interface. \\\hline
    0x40 & PD presents an ultra-high-frequency credential interface (UHF/RAIN RFID). \\\hline
    0x80 & PD supports Virtual Contact Interface (VCI) secure messaging for applicable modes. \\\hline
  \end{tabular}
\end{center}
\end{samepage}
\noindent All other bits are reserved for future use and \textbf{SHALL} be transmitted as zero. PDs \textbf{SHALL} set Bit~\texttt{0x04} when they implement the OSDP~2.3 PIV Mode TLV format and the associated automated flows described in Section~\ref{sec:pivmode}.

\subsubsection{Function Code 14 – Biometric Support}

Function Code~14 remains backward compatible with OSDP~2.2. A compliance value of \texttt{0x00}, \texttt{0x01}, \texttt{0x02}, or \texttt{0x03} retains its original meaning and indicates that the PD does not support the OSDP~2.3 biometric command set. Devices that support the enhanced commands \textbf{SHALL} set Bit~7 (\texttt{0x80}). When Bit~7 is asserted, the remaining bits of the compliance byte are interpreted as a capability bitfield. Bits not explicitly defined in Table~\ref{tab:function-code-14-flags} are reserved and \textbf{SHALL} be transmitted as zero.

\begin{samepage}
\begin{center}
  \captionsetup{type=table}
  \captionof{table}{Function Code 14 Compliance Flags}
  \label{tab:function-code-14-flags}
  \begin{tabular}{|c|p{9cm}|}
    \hline
    \rowcolor{PacketHeaderBg}
    \textcolor{PacketHeaderFg}{\bfseries Mask} & \textcolor{PacketHeaderFg}{\bfseries Meaning} \\\hline
    0x80 & PD supports the OSDP~2.3 biometric command set. \\\hline
    0x01 & PD performs off-card fingerprint comparison using FIPS~201 BIO ANSI INCITS~378 templates. \\\hline
    0x02 & PD performs on-card fingerprint comparison using FIPS~201 OCC processes. \\\hline
    0x04 & PD performs off-card iris comparison aligned with FIPS~201 BIO guidance. \\\hline
    0x08 & PD performs off-card facial comparison using ANSI INCITS~385-2004 images stored on the credential. \\\hline
    0x10 & PD supports OSDP~2.3 autonomous PIV operation (PIV Auto). \\\hline
  \end{tabular}
\end{center}
\end{samepage}

