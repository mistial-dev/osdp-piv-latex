\section{New or Enhanced PIV Commands}

This section introduces commands defined exclusively for the OSDP 2.3 Enhanced PIV profile.

\subsection{PIV Put Data (osdp\_PIVPUTDATA)}

The \texttt{osdp\_PIVPUTDATA} command instructs the PD to deliver an ISO~7816-4 \texttt{PUT DATA} operation to the credential. The ACU \textbf{SHALL} construct the payload as a complete BER-TLV object, including the desired outer tag (\texttt{0x7E} for the Discovery Object, \texttt{0x7F61} for a BIT Group Template, \texttt{0x5C}/\texttt{0x53} sequences for other data objects, and so on). The PD \textbf{SHALL} forward the payload to the credential using \texttt{CLA = 0x00}, \texttt{INS = 0xDB}, \texttt{P1 = 0x3F}, and \texttt{P2 = 0xFF}. If VCI secure messaging is established, the PD \textbf{SHALL} select the secure channel automatically and adjust the \texttt{CLA} accordingly. Chained APDUs \textbf{SHALL} be generated by the PD when the payload exceeds a single-card APDU.

\begin{PacketTable}{PIV Put Data (osdp\_PIVPUTDATA)}
  \PacketRow{1}{CMND}{Command identifier.}{TBA}
  \PacketRow{2}{MpSizeTotal}{Total size of the complete PUT DATA payload, least-significant byte first.}{0x0000--0xFFFF}
  \PacketRow{2}{MpOffset}{Offset of this fragment within the complete payload, least-significant byte first.}{0x0000--0xFFFF}
  \PacketRow{2}{MpFragmentSize}{Number of payload bytes carried in this fragment, least-significant byte first.}{0x0000--0xFFFF}
  \PacketRow{0--n}{Data}{BER-TLV payload to be written to the credential. This field already includes the outer object tag selected by the ACU.}{0x00--0xFF}
\end{PacketTable}


\paragraph{Responses} A successful write is acknowledged with \texttt{osdp\_ACK}. Failures \textbf{SHALL} use \texttt{osdp\_NAK} with one of the Enhanced PIV error codes defined in Subsection~\ref{sec:new-piv-error-codes}: \texttt{0x0A} when the function is not supported, \texttt{0x0B} when the PD lacks buffer space to stage the payload, \texttt{0x0C} when the credential has insufficient storage, or \texttt{0x0D} when security conditions on the credential are not met.

\subsection{Card Status (osdp\_CARDSTATUS)}

The \texttt{osdp\_CARDSTATUS} command requests the current credential context from the PD. The command code is to be assigned; PDs \textbf{SHALL} advertise support via Function Code~13 bit~2 before controllers attempt to use this command. The payload selects the status record that should be returned (currently only the standard profile).

\begin{PacketTable}{Card Status Request (osdp\_CARDSTATUS)}
  \PacketRow{1}{CMND}{Command identifier.}{TBA}
  \PacketRow{1}{Status Type}{Requested status record. \texttt{0x01} selects the standard PIV card status profile. Values \texttt{0x02}--\texttt{0x7F} are reserved and shall elicit \texttt{osdp\_NAK}. Values \texttt{0x80}--\texttt{0xFF} are reserved for private use.}{0x00--0xFF}
\end{PacketTable}


PDs \textbf{SHALL} return \texttt{osdp\_NAK} with an appropriate error code (for example \texttt{0x03} unknown command, \texttt{0x05} incorrect parameter) if the requested status type is not supported.

\subsection{Card Status Reply (osdp\_CARDSTATUSR)}

In response to \texttt{osdp\_CARDSTATUS}, or when the credential state changes after a PIV mode has been selected, the PD reports the detected credential type, VCI state, and optional application data. When the card state changes asynchronously, the PD \textbf{MAY} send \texttt{osdp\_CARDSTATUSR} as the next \texttt{osdp\_POLL} reply following OSDP’s standard unsolicited-report rules. ACUs may therefore receive the reply either immediately after issuing \texttt{osdp\_CARDSTATUS} or as an unsolicited poll response when a credential is presented or removed.

\begin{PacketTable}{Card Status Reply (osdp\_CARDSTATUSR)}
  \PacketRow{1}{CMND}{Reply identifier.}{TBA}
  \PacketRow{1}{Detected Credential Type}{Current card interface: \texttt{0x00} none, \texttt{0x01} ISO~7816 contact, \texttt{0x02} ISO~14443 contactless, \texttt{0x03} other ISO~7816 credential (for example DESFire, Seos), \texttt{0x04} low-frequency credential, \texttt{0x05} unknown type, \texttt{0x06}--\texttt{0x7F} reserved, \texttt{0x80}--\texttt{0xFF} private use.}{0x00--0xFF}
  \PacketRow{1}{VCI Status}{\texttt{0x00} VCI not established, \texttt{0x01} VCI established. Values \texttt{0x02}--\texttt{0x7F} reserved, \texttt{0x80}--\texttt{0xFF} private use.}{0x00--0xFF}
  \PacketRow{1}{VCI Trust Anchor ID}{Identifier of the trust anchor used to establish VCI. \texttt{0x00} indicates no anchor in use.}{0x00--0xFF}
  \PacketRow{2}{PIN Usage Policy}{PIN Usage Policy bytes (tag \texttt{0x5F2F}) returned verbatim when available. A value of \texttt{0xFFFF} indicates the policy is not present or has not yet been read.}{0x0000--0xFFFF}
  \PacketRow{1}{Selected AID Length}{Length (0--16) of the selected AID. \texttt{0x00} indicates no application is currently selected.}{0x00--0x10}
  \PacketRow{0--16}{Selected AID}{Application Identifier read from the credential (for example from the FCI template). Present only when the length is non-zero.}{0x00--0xFF}
\end{PacketTable}


A PD that cannot provide the requested status \textbf{SHALL} return \texttt{osdp\_NAK} and omit the reply. Successful replies include the entire selected AID (when available) and any PIN Usage Policy bytes read from the credential’s Discovery Object (tag \texttt{0x5F2F}). When no PIV PIN Usage Policy is present, the PD \textbf{SHALL} report a length of zero and omit the bytes. The VCI fields describe whether a secure channel is currently established, which trust anchor enabled the session, and whether the presented credential requires a pairing code for VCI operation.

\subsection{TWIC Privacy Key Load (osdp\_PIV\_LOADTPK)}

The \texttt{osdp\_PIV\_LOADTPK} command instructs the PD to cache a TWIC Privacy Key (TPK) for use with subsequent biometric operations against a TWIC credential. The command code is to be assigned. PDs \textbf{SHALL} advertise support for TWIC biometric processing via Function Code~14 before controllers attempt to use this command.

\begin{PacketTable}{TWIC Privacy Key Load (osdp\_PIV\_LOADTPK)}
  \PacketRow{1}{CMND}{Command identifier.}{TBA}
  \PacketRow{1}{Cache Timeout}{Number of seconds to retain the cached TPK. \texttt{0x00} clears any cached key immediately; \texttt{0x01}--\texttt{0xFF} permit caching for up to 255 seconds.}{0x00--0xFF}
  \PacketRow{1}{Control Flags}{Bit~0 set requests automatic clearing when the credential is removed; all other bits \textbf{SHALL} be zero.}{0x00--0xFF}
  \PacketRow{32}{TPK}{Trusted Processing Key bytes to cache for the next TWIC biometric operation.}{0x00--0xFF}
\end{PacketTable}


Controllers set \emph{Cache Timeout} to define how long (in seconds) the PD may retain the key. A value of \texttt{0x00} clears any cached TPK immediately. Values \texttt{0x01}--\texttt{0xFF} allow caching for up to 255 seconds. Bit~0 of \emph{Control Flags} directs the PD to clear the cached TPK when the credential is removed; all other bits are reserved and \textbf{SHALL} be transmitted as zero. If the PD does not support TWIC TPK caching it \textbf{SHALL} return \texttt{osdp\_NAK} \texttt{0x03}. Invalid payloads (for example reserved bits set) \textbf{SHALL} elicit \texttt{osdp\_NAK} \texttt{0x05}.

\subsection{PIV VCI Pairing Code Transmit (osdp\_PIV\_XMITPAIRING)}

The \texttt{osdp\_PIV\_XMITPAIRING} command delivers an eight-digit pairing code to the PD for use during Virtual Contact Interface (VCI) establishment. The command code is to be assigned. Controllers \textbf{SHALL NOT} issue this command unless a PIV application is currently selected and VCI mode is active.

\begin{PacketTable}{PIV VCI Pairing Code Transmit (osdp\_PIV\_XMITPAIRING)}
  \PacketRow{1}{CMND}{Command identifier.}{TBA}
  \PacketRow{8}{Pairing Code}{Eight ASCII digits (\texttt{'0'}--\texttt{'9'}) representing the VCI pairing code associated with the currently selected PIV application.}{0x30--0x39}
\end{PacketTable}


The pairing code is encoded as eight ASCII digits (\texttt{'0'}--\texttt{'9'}). A PD that has not yet established VCI \textbf{SHALL} return \texttt{osdp\_NAK} \texttt{0x0F} (VCI not established). If no PIV application is selected, the PD \textbf{SHALL} return \texttt{osdp\_NAK} \texttt{0x05}. PDs \textbf{SHALL} clear the stored pairing code when VCI is torn down or when the active PIV mode changes.

\subsection{Set PIV Mode (osdp\_PIVMODE)}\label{sec:pivmode}

The \texttt{osdp\_PIVMODE} command defines the operating context for PIV, TWIC, and related credential transactions. OSDP~2.3 replaces the legacy “Application + AID” fields with a TLV list of configuration descriptors so controllers can specify multiple application profiles, preferred interfaces, and optional Application Identifiers. The command code remains to be assigned. PDs \textbf{SHALL} advertise support via Function Code~13 bit~\texttt{0x04}.

\begin{PacketTable}{Set PIV Mode (osdp\_PIVMODE)}
  \PacketRow{1}{CMND}{Command identifier.}{TBA}
  \PacketRow{1}{PIV Mode Flags}{Global behaviour flags (Section~\ref{sec:pivmode}).}{0x00--0xFF}
  \PacketRow{2}{MpSizeTotal}{Total size of the complete configuration payload, least-significant byte first.}{0x0000--0xFFFF}
  \PacketRow{2}{MpOffset}{Offset of this fragment within the configuration payload, least-significant byte first.}{0x0000--0xFFFF}
  \PacketRow{2}{MpFragmentSize}{Length of the fragment payload, least-significant byte first.}{0x0000--0xFFFF}
  \PacketRow{1}{Config Count}{Number of configuration entries (1--32).}{0x01--0x20}
  \PacketRow{0--n}{TLV Data}{Sequence of configuration descriptors: optional Global Interface Mask TLV (Tag \texttt{0x10}) followed by one or more Configuration Entry TLVs (Tag \texttt{0x01}).}{--}
\end{PacketTable}


The \emph{PIV Mode Flags} byte enables global behaviours such as autonomous operation and VCI usage (Table~\ref{tab:pivmode-flags}). Bits not listed remain reserved and \textbf{SHALL} be transmitted as zero.

\Needspace{12\baselineskip}
\begin{samepage}
\begin{center}
  \captionsetup{type=table}
  \captionof{table}{PIV Mode Flags}
  \label{tab:pivmode-flags}
  \begin{tabular}{|c|p{9cm}|}
    \hline
    \rowcolor{PacketHeaderBg}
    \textcolor{PacketHeaderFg}{\bfseries Mask} & \textcolor{PacketHeaderFg}{\bfseries Meaning} \\\hline
    0x01 & Enable autonomous PIV (PIV Auto). The PD interrogates credentials, establishes VCI when possible, and advances through the authentication flow without explicit ACU prompts. \\\hline
    0x02 & Permit VCI usage when available. \\\hline
    0x04 & Enforce FICAM strict mode (PIV applications only). \\\hline
  \end{tabular}
\end{center}
\end{samepage}


After the flags, the payload uses standard multi-part framing fields and a configuration count (1--32). Each configuration appears as a TLV:

\begin{itemize}
  \item \textbf{Global Interface Mask (Tag \texttt{0x10}, optional)} — One-byte mask that limits the interfaces the PD may use for all configurations. Bit assignments follow Function Code~13 (bit~0 contact, bit~1 contactless, bit~2 low-frequency, bit~3 Bluetooth, bit~4 barcode, bit~5 UHF/RAIN). A value of \texttt{0x00} indicates no global restriction.
  \item \textbf{Configuration Entry (Tag \texttt{0x01})} — Contains the data-model identifier, a per-entry interface mask, the AID length, and the optional AID:
    \begin{itemize}
      \item \emph{Data Model} (\texttt{0x01} PIV, \texttt{0x02} TWIC, values \texttt{0x04}--\texttt{0x7F} reserved, \texttt{0x80}--\texttt{0xFF} private use).
      \item \emph{Interface Mask} — Uses the same bit positions as the global mask; \texttt{0x00} means “any interface permitted by the global mask.”
      \item \emph{AID Length} (0--16) and \emph{AID} bytes when provided.
    \end{itemize}
\end{itemize}

PDs process configuration entries in order, applying the global mask (when present) before honouring each entry’s mask. Invalid masks, unknown data-model identifiers, or malformed TLVs \textbf{SHALL} result in \texttt{osdp\_NAK} \texttt{0x05}. Controllers \textbf{SHALL} send the command over a secure channel where required; the PD \textbf{SHALL} return \texttt{osdp\_NAK} \texttt{0x06} if the necessary security conditions are not satisfied. Multi-part framing errors result in \texttt{osdp\_NAK} \texttt{0x07}. If the PD cannot cache the requested configuration set, it \textbf{SHALL} respond with \texttt{osdp\_NAK} \texttt{0x0B}.

When PIV Auto is enabled, PDs interrogate the credential on presentation, establish VCI automatically when possible, and advance through authentication without explicit ACU prompts. If a pairing code is required, the PD signals the state change via \texttt{osdp\_CARDSTATUSR}; the ACU then provides the code using \texttt{osdp\_PIV\_XMITPAIRING}. Commands that rely on PIV Auto \textbf{SHALL} return \texttt{osdp\_NAK} \texttt{0x10} when the feature has not been configured.

\subsection{VCI Trust Anchor Load (osdp\_PIV\_VCILOADTA)}

The \texttt{osdp\_PIV\_VCILOADTA} command loads, replaces, or deletes a Virtual Contact Interface (VCI) trust anchor. Trust anchors must be cached before a PD can verify card certificates during VCI establishment. The command code is to be assigned.

\begin{PacketTable}{VCI Trust Anchor Load (osdp\_PIV\_VCILOADTA)}
  \PacketRow{1}{CMND}{Command identifier.}{TBA}
  \PacketRow{1}{Operation}{\texttt{0x00} load/replace anchor; \texttt{0x01} delete anchor; \texttt{0x02} query anchor inventory.}{0x00--0x02}
  \PacketRow{1}{Trust Anchor ID}{Controller-assigned identifier for the anchor (1--128).}{0x00--0xFF}
  \PacketRow{8}{Issuer Identifier Number}{Leftmost eight bytes of the subjectKeyIdentifier from the issuing content-signing certificate.}{0x00--0xFF}
  \PacketRow{2}{Anchor Length}{Total size of the anchor payload, least-significant byte first. Set to \texttt{0x0000} when deleting or querying inventory.}{0x0000--0xFFFF}
  \PacketRow{2}{MpSizeTotal}{Total size of the complete anchor data, least-significant byte first. Set to \texttt{0x0000} when deleting or querying inventory.}{0x0000--0xFFFF}
  \PacketRow{2}{MpOffset}{Offset of this fragment within the anchor data, least-significant byte first. Set to \texttt{0x0000} when deleting or querying inventory.}{0x0000--0xFFFF}
  \PacketRow{2}{MpFragmentSize}{Length of the data block in this fragment, least-significant byte first. Set to \texttt{0x0000} when deleting or querying inventory.}{0x0000--0xFFFF}
  \PacketRow{0--n}{Anchor Data}{Anchor bytes (DER or equivalent) present when loading. Omitted for delete and query operations. Subsequent fragments (load only) contain only this field.}{0x00--0xFF}
\end{PacketTable}


The metadata block (operation, identifier, IIN, anchor length) appears only in the first fragment where \texttt{MpOffset = 0}; subsequent fragments carry additional anchor bytes. Controllers set \emph{Operation} to \texttt{0x00} to load or replace the anchor identified by \emph{Trust Anchor ID}. Set \emph{Operation} to \texttt{0x01} with \emph{Anchor Length} \texttt{0x0000} to delete the anchor. The \emph{Issuer Identifier Number (IIN)} is the leftmost eight bytes of the subjectKeyIdentifier in the content signing certificate used to protect the credential. The \emph{Anchor Length} declares the total size of the anchor payload; controllers shall send the exact number of bytes in one or more fragments.

PDs \textbf{SHALL} reconstruct the anchor from the fragments before storing it. A successful operation returns \texttt{osdp\_ACK}. Failures return \texttt{osdp\_NAK} with:
\begin{itemize}
  \item \texttt{0x03} when the PD does not support trust-anchor caching.
  \item \texttt{0x05} for malformed metadata (unknown operation, reserved bits, invalid lengths).
  \item \texttt{0x07} when multi-part sequencing fails.
  \item \texttt{0x0B} when the PD lacks storage space for the anchor.
\end{itemize}

PDs are not required to maintain a real-time clock; they need only verify that credential certificates presented later fall within the trust anchor’s validity period. Implementers may perform stricter date checks if desired. PDs may retain either the full anchor blob or a compact record containing the public key, issuer identifier, validity period, and anchor number so long as the supplied IIN remains available for matching.

\subsection{VCI Trust Anchor Status (osdp\_PIV\_VCITASTATUS)}

The \texttt{osdp\_PIV\_VCITASTATUS} command enumerates cached trust anchors and reports remaining storage capacity. The command code is to be assigned. The request carries no payload. The reply uses the standard multi-part fields (\texttt{TOTAL}, \texttt{OFFSET}, \texttt{DATA\_LEN}) so PDs can stream the inventory across one or more fragments.

\begin{PacketTable}{VCI Trust Anchor Status Reply (osdp\_PIV\_VCITASTATUSR)}
  \PacketRow{1}{CMND}{Reply identifier.}{TBA}
  \PacketRow{2}{TOTAL}{Length of the complete status payload, least-significant byte first.}{0x0000--0xFFFF}
  \PacketRow{2}{OFFSET}{Offset of this fragment within the status payload, least-significant byte first.}{0x0000--0xFFFF}
  \PacketRow{2}{DATA\_LEN}{Length of the data block in this fragment, least-significant byte first.}{0x0000--0xFFFF}
  \PacketRow{1}{Total Anchor Count}{Number of cached trust anchors.}{0x00--0xFF}
  \PacketRow{2}{Available Bytes}{Remaining storage capacity for trust anchors, least-significant byte first.}{0x0000--0xFFFF}
  \PacketRow{0--n}{Anchor Records}{Zero or more records. Each record contains: \texttt{Trust Anchor ID} (1 byte), \texttt{Issuer Identifier Number} (8 bytes), \texttt{Anchor Length} (2 bytes LSB first), \texttt{Attributes} (1 byte, reserved). Records that span fragments continue in subsequent replies.}{--}
\end{PacketTable}


The PD first reports the total number of anchors and the remaining free bytes. Each record thereafter lists a trust anchor by ID, IIN, anchor length, and reserved attributes (transmitted as zero). Controllers can use this information to decide whether to load additional anchors or delete existing ones. NAK responses follow the standard pattern: \texttt{0x03} for unsupported command, \texttt{0x07} for multi-part errors.
